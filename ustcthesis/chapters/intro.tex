\chapter{引言}
\section{编写目的}
编写本说明书的目的在于明确“拍吧”摄影平台开发的的功能和性能需求,以标准的语言和表述方式整理系统需求,明确并规范各模块接口,以便于开发者和用户协同明确开发目标,让程序设计员能加强对本平台的理解,为其提供一个在设计和以后的软件测试以及软件维护的参考。

\section{项目背景}
伴随经济的发展、社会的进步,智能手机与专业相机迅速普及,人们有着更多的途径接触摄影,越来越多的人开始学习摄影并逐渐对其产生浓厚的兴趣,摄影已经成为了人们生活中重要的一部分。人们开始习惯于美餐一顿后发朋友圈炫耀,游览名胜时拍照片留下美好的回忆,甚至在“优秀男朋友特质”中也加入了“能随时为女朋友拍出美丽的照片”这一条。摄影,已经成为了一种分享生活点滴的重要形式、一种对人生重要时刻记录的重要方式、一种休闲及交流良好平台。然而,要想拍出真正高质量的照片绝非易事,它不仅仅需要一台性能优良的手机或相机、简单的美颜与滤镜功能,更需要拍摄时对角度、光线的选择,取景、构图、后期处理等多个复杂的因素共同作用,这就为摄影爱好者们想要拍摄出自己期望的效果造成了很大困难。同时,那些真正的职业摄影师、有着较高摄影技巧的人们也缺少一个合适的平台进行交流,无法将自己的经验技巧传授给大家,这使得摄影爱好者们难以入门。在此背景下,我们的项目“拍吧”摄影平台应运而生。它是一个独立的、目前尚未出现功能完全一致的新系统,运行在Android手机端,作用是为摄影爱好者们提供一个专门的平台进行交流,使更多的人能够进一步了解摄影、爱上摄影。本项目由中国科学技术大学Android开发小组承包开发,由中国科学技术大学XX摄影协会委托。

\section{术语}
\begin{table}[htbp]
\centering
\caption{术语表} \label{tab:terminology}
\begin{tabular}{|c|c|}
    \hline
    缩写、术语 & 解释 \\
    \hline
    用户User & 本平台的客户端使用者 \\
    \hline
	管理者Administrator & 本平台中服务器管理者 \\
    \hline
\end{tabular}
% \note{这里是表的注释}
\end{table}