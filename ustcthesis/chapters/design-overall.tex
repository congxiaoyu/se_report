\chapter{总体设计}
\section{软件描述}
系统包括客户端和服务器端两个部分。

客户端主要功能是:提供界面方便用户对本软件的使用及进行交互

服务器端主要功能是:接受用户数据请求并对数据库进行查询、更新等操作,并将操作结果返回到客户端

\section{处理流程}
\subsection{总体流程}
客户端输入数据并发送请求,服务器端接收请求并执行相应数据库操作,然后将操作结果返回到客户端。

\subsection{客户端基本流程}
先进行输入操作(某些功能不需要进行输入),输入完毕后点击确认按钮,客户端会根据输入内容将输入打包为JSON字符串发送到服务器端,等待服务器端进行处理后接收返回结果并显示给用户。

\subsection{服务器端基本流程}
接收客户端发送的数据及请求,根据请求内容进行数据库相应操作(更新、查询等),并将操作结果(更新是否成功、查询结果)打包发送回客户端。

\subsection{用户注册具体流程}
用户在首次使用本软件时,输入用户名、邮件、手机、密码、确认密码,点击注册按钮后客户端会先对输入的合法性(包括手机号码的合法性、邮箱的合法性等)进行检测,检测通过后产生一个JSON字符串发送到服务器端,服务器端进行重复注册的合法性检测,检测通过后会将注册确认邮件发送到用户邮箱,用户通过登录邮箱进行注册确认。


\subsection{用户登录具体流程}
用户输入用户名、密码,点击登录按钮后客户端会先对输入数据合法性进行检测,检测通过后将输入产生一个JSON字符串发送到服务器端,服务器端进行密码正确性检测,检测通过后将消息发送回客户端,客户端收到成功消息后跳转到用户主界面。

\subsection{用户搜索用户具体流程}
用户输入待搜索用户名,点击搜索按钮后客户端将输入结果产生一个JSON字符串发送到服务器端,服务器端解析后在数据库进行查询,并将查询结果返回给客户端,客户端接收后解析,分条展示给用户。

\subsection{用户信息查看具体流程}
用户点击其他用户或自己的基本信息栏,客户端将双方部分信息发送给服务器,服务器在数据库中进行查询,结合待搜索用户信息开放度将结果发送回客户端,客户端进行解析后将结果展示给用户。

\subsection{用户信息更新具体流程}
用户输入待更改的信息,客户端将修改内容发送到服务器,服务器进行修改结果合法性检测,检测通过后会进行数据库更新操作,成功后将修改后的信息发送回客户端,客户端接收并解析后更新用户界面。

\subsection{用户好友会话具体流程}
用户输入要发送的消息(包括图片消息、语音消息等),点击发送按钮后客户端将输入产生一个JSON字符串发送到服务器端,服务器端进行合法性检测(包括输入是否为空及是否含有敏感词等),检测成功后将消息发送到接受者客户端,并将发送成功消息发送到发送者客户端。

\subsection{用户好友添加具体流程}
用户进入其他用户个人主页后点击添加好友按钮,客户端生成请求并发送到服务器端,服务器接收请求后将其发送到被申请者客户端,被申请者点击同意或拒绝按钮后将回应发送到服务器,服务器收到回应后将结果返回给申请者客户端,并根据回应决定是否进行数据库更新操作,申请者客户端接收结果后进行界面更新。

\subsection{用户好友删除具体流程}
用户进入已添加好友的个人主页后点击删除好友按钮,客户端生成请求并发送到服务器端,服务器接收请求后进行数据库更新操作,操作成功后将成功消息发送回申请者客户端,申请者客户端接收消息后进行界面更新。

\subsection{用户举报用户具体流程}
用户进入其他用户个人主页并点击举报按钮,输入举报理由,点击举报按钮后客户端将输入产生一个JSON字符串发送到服务器端,服务器端接收后通知系统管理者进行审核,并将回执消息发送回客户端,客户端接收回执消息后更新界面。

\subsection{服务器自动推荐文章并显示文章列表具体流程}
用户进入主界面或在主界面下拉更新时客户端产生推荐请求并发送到服务器,服务器端使用推荐算法找出一定数量文章并将这些文章的信息打包为JSON字符串发送回客户端,客户端进行解析后分条展示给用户。

\subsection{用户热搜具体流程}
用户进入到搜索界面时客户端产生热搜请求并发送到服务器,服务器端收到请求后将当前热搜产生一个JSON字符串发送回客户端,客户端进行解析后分条展示给用户。

\subsection{用户搜索文章具体流程}
用户输入待搜索文章关键字以及搜索限制选项,客户端将输入产生一个JSON字符串发送到服务器端,服务器端进行解析后到数据库中进行查询,并将查询结果返回给客户端,客户端进行解析后分条展示给用户。

\subsection{用户发表文章具体流程}
用户输入文章标题、内容、图片等,点击发表按钮,客户端将输入打包后发送到服务器端,服务器端进行合法性检测(包括长度检测、敏感词检测等),检测通过后进行数据库的更新,并将更新结果返回给客户端,客户端接收更新成功消息后进行界面跳转。

\subsection{用户编辑已发表文章具体流程}
用户对修改自己发表的文章标题、内容、图片等,点击发表按钮,客户端将输入打包后发送到服务器端,服务器端进行合法性检测(包括长度检测、敏感词检测等),检测通过后进行数据库的更新,并将更新结果返回给客户端,客户端接收更新成功消息后进行界面跳转。

\subsection{用户阅读文章详细内容具体流程}
用户点击文章缩略图,客户端根据点击位置产生不同的请求发送到服务器端,服务器根据请求内容进行数据库查询操作,将查询结果返回给客户端,客户端根据不同返回结果将内容展示给用户。

\subsection{用户发表评论具体流程}
用户输入评论内容,点击评论按钮后客户端将评论内容产生一个JSON字符串发送到服务器端,服务器进行合法性检测(敏感词检测等)后进行数据库更新操作,并将更新成功消息发送回客户端,客户端接收消息后更新界面。

\subsection{用户删除文章评论具体流程}
用户对自己发表过的评论点击删除按钮,客户端产生一个请求发送到服务器端,服务器端接收请求后进行数据库更新操作,并将更新成功消息发送回客户端,客户端接收消息后更新界面。

\subsection{用户获取文章评论具体流程}
用户对某个文章阅读时点击查看更多评论按钮,客户端将文章信息及请求打包发送到服务器端,服务器端进行解析后进行数据库查询操作,并将查询结果返回到客户端,客户端进行解析后分条展示给用户。

\subsection{用户点赞具体流程}
用户对某个文章阅读时点击赞同按钮,客户端将点赞请求发送到服务器端,服务器端接收请求并进行数据库更新操作,并将更新成功消息发送回客户端,客户端接收消息后更新界面。

\subsection{用户取消点赞具体流程}
用户对某个已点赞文章进行取消点赞,客户端将取消点赞请求发送到服务器端,服务器端接收请求并进行数据库更新操作,并将更新成功消息发送回客户端,客户端接收消息后更新界面。

\subsection{用户举报文章具体流程}
用户阅读文章时点击举报按钮,输入举报理由,点击举报按钮后客户端将输入产生一个JSON字符串发送到服务器端,服务器端接收后通知系统管理者进行审核,并将回执消息发送回客户端,客户端接收回执消息后更新界面。

\subsection{用户照片墙具体流程}
用户通过个人主页进入照片墙,客户端产生照片查询请求发送到服务器端,服务器端接收请求并在数据库中进行查询操作,并将查询结果返回到客户端,客户端进行解析后将每个照片显示在用户终端。

\subsection{管理员处理举报具体流程}
管理员在收到举报后对被举报文章或用户进行数据查询并进行审核,并输入审核结果发送到服务器端,服务器端接收审核结果后进行相应的删除操作等,并将审核结果发送到举报者和被举报者客户端,举报者客户端接收审核结果并更新界面。

\section{功能结构设计}
\subsection{整体结构}
此处应当有一个图和对应的描述。系统如果像微内核那样,划分成核心模块和若干个子系统,此处应当有图示及说明,然后后续几个节应当描述这几个子系统。如果系统像宏内核,那应当说明有哪些紧密联系的模块,并在后续几个节内描述这些模块。

\subsection{用户端结构}
此处应当有一个图和对应的描述。这只是举个例子。可能的内容包括用户端的具体模块、耦合情况等。

\subsection{服务器端结构}
此处应当有一个图和对应的描述。这只是举个例子。

\subsection{后台数据库维护模块结构}
此处应当有一个图和对应的描述。这只是举个例子。



\section{功能需求与程序代码的关系}
[此处指的是不同的需求分配到哪些模块去实现。可按不同的端拆分此表]
\begin{table}[htbp]
\centering
\caption{功能需求与程序代码的关系表} \label{tab:requirement-module}
\begin{tabular}{|c|c|c|c|}
    \hline
    · & 模块1 & 模块2 & 模块3 \\
    \hline
    需求1 & · & Y & · \\
    \hline
    需求2 & · & Y & · \\
    \hline
    需求3 & · & Y & · \\
    \hline
    需求4 & Y & · & · \\
    \hline
    需求5 & · & · & Y \\
    \hline
\end{tabular}
\note{各项功能需求的实现与各个程序模块的分配关系}
\end{table}
