\chapter{总体设计}
\section{软件描述}
系统包括前台和后台两个部分。

前台主要功能是:提供界面方便用户对本软件的使用及进行交互

后台主要功能是:接受用户数据请求并对数据库进行查询、更新等操作,并将操作结果返回到前台

\section{处理流程}
\subsection{总体流程}


\subsection{系统基本流程}
此处应当有一个图和对应的描述。

\subsection{客户端基本流程}
先进行输入操作(某些功能不需要进行输入),输入完毕后点击确认按钮,

\subsection{服务器端基本流程}
这只是举个例子,如果没有服务器端则不需要此节。

\subsection{用户注册具体流程}
用户在首次使用本软件时,输入用户名、邮件、手机、密码、确认密码,点击注册按钮后客户端会先对输入的合法性(包括手机号码的合法性、邮箱的合法性等)进行检测,检测通过后产生一个JSON字符串发送到服务器端,服务器端进行重复注册的合法性检测,检测通过后会将注册确认邮件发送到用户邮箱,用户通过登录邮箱进行注册确认。


\subsection{用户登录具体流程}
用户输入用户名、密码,点击登录按钮后客户端会先对输入数据合法性进行检测,检测通过后将输入产生一个JSON字符串发送到服务器端,服务器端进行密码正确性检测,检测通过后将消息发送回客户端,客户端收到成功消息后跳转到用户主界面。

\subsection{用户搜索具体流程}
用户输入待搜索用户名,点击输入按钮后客户端将输入结果产生一个JSON字符串发送到服务器端,服务器端解析后在数据库进行查询,并将查询结果返回给客户端,客户端接收后解析,分条展示给用户。

\subsection{用户信息查看具体流程}
用户点击其他用户或自己的基本信息栏,客户端将双方部分信息发送给服务器,服务器在数据库中进行查询,结合待搜索用户信息开放度将结果发送回客户端,客户端进行解析后将结果展示给用户。

\subsection{用户信息更新具体流程}
用户输入待更改的信息,客户端将修改内容发送到服务器,服务器进行修改结果合法性检测,检测通过后会进行数据库更新操作,成功后将修改后的信息发送回客户端,客户端接收并解析后更新用户界面。

\subsection{用户好友会话具体流程}
用户输入要发送的消息(包括图片消息、语音消息等),点击发送按钮后客户端将输入产生一个JSON字符串发送到服务器端,服务器端进行合法性检测(包括输入是否为空及是否含有敏感词等),检测成功后将消息发送到接受者客户端,并将发送成功消息发送到发送者客户端。

\subsection{用户好友添加具体流程}
用户进入其他用户个人主页后点击添加好友按钮,客户端生成请求并发送到服务器端,服务器接收请求后将其发送到被申请者客户端,被申请者点击同意或拒绝按钮后将回应发送到服务器,服务器收到

\section{功能结构设计}
\subsection{整体结构}
此处应当有一个图和对应的描述。系统如果像微内核那样,划分成核心模块和若干个子系统,此处应当有图示及说明,然后后续几个节应当描述这几个子系统。如果系统像宏内核,那应当说明有哪些紧密联系的模块,并在后续几个节内描述这些模块。

\subsection{用户端结构}
此处应当有一个图和对应的描述。这只是举个例子。可能的内容包括用户端的具体模块、耦合情况等。

\subsection{服务器端结构}
此处应当有一个图和对应的描述。这只是举个例子。

\subsection{后台数据库维护模块结构}
此处应当有一个图和对应的描述。这只是举个例子。



\section{功能需求与程序代码的关系}
[此处指的是不同的需求分配到哪些模块去实现。可按不同的端拆分此表]
\begin{table}[htbp]
\centering
\caption{功能需求与程序代码的关系表} \label{tab:requirement-module}
\begin{tabular}{|c|c|c|c|}
    \hline
    · & 模块1 & 模块2 & 模块3 \\
    \hline
    需求1 & · & Y & · \\
    \hline
    需求2 & · & Y & · \\
    \hline
    需求3 & · & Y & · \\
    \hline
    需求4 & Y & · & · \\
    \hline
    需求5 & · & · & Y \\
    \hline
\end{tabular}
\note{各项功能需求的实现与各个程序模块的分配关系}
\end{table}